
This thesis explores contemplative practices and cognitive neuroscience through machine learning applied to EEG data, laying the groundwork for developing a methodological framework assessing mind wandering. With a growing body of research underscoring the benefits of practices such as meditation on mental health, there is a critical need for objective tools to measure their impact effectively.

Leveraging a dataset that includes EEG recordings from tasks designed to induce mind wandering, this work focuses on building and refining a Convolutional Neural Network (CNN) based on the EEGNeX architecture. The primary goal is to classify states of attention and mind wandering, thereby providing a tool to assess the efficacy and progression of contemplative practices. This classification aids in understanding how these practices influence cognitive processes and mental states, which can be particularly useful for tailoring individual therapeutic approaches in clinical settings.

In developing this CNN, various data balancing techniques were applied to address class imbalances, and noise reduction strategies were employed to enhance the quality of EEG signals. The model was trained and validated using rigorous cross-validation techniques to ensure its robustness and generalizability across different individuals and contemplative states using two different mind-wandering-inducing tasks.

The findings from this thesis could contribute to contemplative neuroscience by offering insights into methods for classifying neural mechanisms underlying mind wandering. This work provides a foundation for future research and the development of objective measures of progress in contemplative practices for possible improvements in mental health.

Ultimately, this thesis underscores the potential of combining traditional contemplative practices with modern neuroscience and machine learning techniques, paving the way for innovative approaches to enhancing well-being and treating mental health conditions.

\newpage
\chapter*{Sammendrag}

Denne Masteroppgaven utforsker kontemplative praksiser gjennom bruk av maskinlæring på EEG-data, med mål om å utvikle et metodisk rammeverk for å vurdere tankevandring (Mind Wandering) og meditative tilstander. Med en økende mengde forskning på feltet som understreker fordelene med øvelser som meditasjon og yoga på mental helse, er det et stort behov for objektive verktøy for måling av deres innvirkning på kropp og sinn.

Ved å utnytte et datasett som inkluderer EEG-opptak fra oppgaver designet for å forårsake tankevandring, fokuserer denne masteroppgaven på å lage og finjustere et konvolusjonsnettverk (CNN) basert på EEGNeX-arkitekturen. Hovedmålet er å klassifisere tilstander av oppmerksomhet og tankevandring, noe som gir et verktøy for å vurdere effektiviteten av, og individers fremgangen i, kontemplative praksiser. Denne klassifiseringen bidrar til å forstå hvordan disse praksisene påvirker kognitive prosesser og mentale tilstander, noe som kan være spesielt nyttig for å skreddersy individuelle terapeutiske tilnærminger.

I prosessen med å utvikle denne CNN modellen ble ulike teknikker for datautjevning anvendt for å takle klasseubalanser, og støyreduksjonsstrategier ble benyttet for å forbedre kvaliteten på EEG-signalene. Modellen ble trent og validert ved hjelp av kryssvalideringsteknikker for å sikre dens robusthet og generaliserbarhet på tvers av ulike individer og kontemplative tilstander ved bruk av to ulike tankevandringsfremmende øvelser.

Funnene fra denne avhandlingen kan bidra  til feltene kontemplativ nevrovitenskap og mental helse, og tilby nye innsikter i de nevrale mekanismene som ligger til grunn for meditasjon og tankevandring. Dette arbeidet gir et grunnlag for videre forskning på området, og har potensial til å veilede utviklingen av målrettede anvendelser av kontemplative praksiser for mental helse og kognitiv forbedring.

Til slutt påpeker denne avhandlingen potensialet for å kombinere tradisjonelle kontemplative praksiser med moderne nevrovitenskap og maskinlæringsteknikker, og baner vei for innovative tilnærminger til å forbedre velvære og behandle psykiske helseproblemer.





