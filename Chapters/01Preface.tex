This thesis is the result of my academic and personal journey into the fields of contemplative practices and cognitive neuroscience. Ever since I started meditating, I've been interested in the topics. My personal experience with meditation has improved my focus and emotional regulation, which sparked my curiosity about the scientific basis of these practices. I believed that there might have been a difference between the popular claims about meditation and the scientific research. I wanted to explore this gap in hopes of grounding my practice in empirical evidence and strengthening my commitment to it.
\newline
Conducting this research has not been without its challenges. The gap in practical alignment between my master's thesis and the preceding project thesis presented a steep learning curve, as it felt like starting from scratch. I found myself navigating the complexities of high-performance computing interfaces and delving into the intricacies of CNNs with limited prior experience. Moreover, working with EEG data introduced a unique set of challenges, from understanding its specific requirements to adapting methodologies that would faithfully capture and interpret the nuances of such data.
\newline
Throughout this process, constructing a scalable and robust code base emerged as a crucial lesson. The flexibility and efficiency of handling large datasets, especially when not initially prepared for the complex analyses planned, became apparent as a foundational skill. Furthermore, the decision to use a dataset not collected by our team introduced additional hurdles in data interpretation and required adaptations to fit our analytical framework.
\newline
Collaboration played a pivotal role in overcoming many of these obstacles. I was fortunate to connect with the lead author of our dataset, Christina Yi Jin, who provided invaluable assistance by clarifying data specifics and supplying missing components essential for our analyses. While this interaction did not directly influence our research outcomes, it enabled us to use the dataset effectively, highlighting the importance of collaboration in scientific endeavors.
\newline
I envision this thesis as a stepping stone toward developing more refined tools for measuring and understanding mind-wandering and meditative states. These tools have immense potential to advance scientific knowledge and provide practical benefits in meditation training. I aspire for future researchers to build upon this work, expanding the dataset, enhancing methodological rigor, and perhaps integrating these tools into systems that offer real-time, personalized guidance for meditators.
\newline
I want to thank my supervisor, Marta Molinas, whose guidance was invaluable throughout this journey. Her expertise and insightful feedback were crucial in shaping the direction and execution of this research. I am profoundly grateful to my girlfriend, Isabella Starrfield, for her support and understanding throughout the ups and downs of this project. Her encouragement has been a constant source of motivation, and without it, this thesis would not be completed on time.
\newline
I am also incredibly thankful to my parents, whose love and support have been foundational throughout my life. Their encouragement and belief in my abilities have always been precious to me.
This thesis is dedicated to all who find meditation a path to greater clarity and peace and the scientific community that seeks to understand how such simple practices can effect profound changes in our minds and bodies. I hope this work contributes meaningfully to both realms, bridging the gap between anecdotal benefits and scientific validation.
\newline
With gratitude,\newline
Erlend Withammer-Ekerhovd