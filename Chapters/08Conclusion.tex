
This thesis has laid the foundation for and explored the impact of various machine learning techniques and parameters on mind-wandering detection using EEG data. Through rigorous experimentation and analysis, it has provided insights into the complexities of EEG data processing and the generalization capabilities of neural networks across different subjects.

Regularization techniques such as L2 and dropout have been shown to be effective in mitigating overfitting. L2 demonstrates particularly robust performance in maintaining model generalization across training, validation, and test datasets. The choice of the regularization parameter played a critical role, with a moderate regularization level proving most effective in balancing model training and generalization.

The experiments with data augmentation, mainly using synthetic data through MEMD augmentation, highlighted the challenges of incorporating synthetic EEG data into training sets. While it was hypothesized that synthetic data would enhance model training by providing more samples, the results indicated otherwise, suggesting that synthetic data might introduce noise or irrelevant patterns that could confuse the learning algorithm. It is speculated that this is caused by data mixing between subjects.

Model performance metrics such as AUROC and Kappa provided a nuanced view of model effectiveness, underscoring the limitations of relying solely on accuracy for evaluating model performance, especially in unbalanced datasets. The application of different metrics revealed that while models could achieve high accuracy, their ability to discern between classes genuinely was not always satisfactory, indicating potential biases or overfitting to dominant classes.

Furthermore, deploying different dropout rates and early stopping techniques underscored the importance of carefully tuned hyperparameters in achieving optimal model performance. These techniques helped fine-tune the models to avoid overfitting while allowing adequate training.

The study also highlighted the significance of dataset characteristics, particularly the balance and distribution of data. The skewed distribution of mind-wandering episodes across subjects suggested potential mislabeling or inherent biases in the dataset, which could impact the training and performance of machine learning models. This calls for a more refined approach to data collection and preprocessing to ensure the reliability and validity of training datasets in EEG-based studies.

In conclusion, this thesis advances our understanding of the practical challenges in using machine learning for EEG data analysis and sets the groundwork for future research. It opens avenues for further exploration into more sophisticated data augmentation techniques, deeper analysis of regularization impacts, and the development of more robust models sensitive to the nuances of EEG data and the phenomenon of mind-wandering. Future studies could also explore the potential of transfer learning and fine-tuning models on individual subjects to enhance prediction accuracy and personalize mind-wandering detection systems, ultimately leading to more effective and user-tailored applications in real-world scenarios.